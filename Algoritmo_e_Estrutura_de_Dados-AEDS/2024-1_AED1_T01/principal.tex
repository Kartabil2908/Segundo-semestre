%%%%%%%%%%%%%%%%%%%%%%%%%%%%%%%%%%%%%%%%%%%%%%%%%%%%%%%%%%%%%%%%%%%%%%%%%%%%%%%%%%%%%%%%%%%%%%%%%%%%%%%
%%%%%%%%%%%%%% Template de Artigo Adaptado para Trabalho de Diplomação do ICEI %%%%%%%%%%%%%%%%%%%%%%%%
%% codificação UTF-8 - Abntex - Latex -  							     %%
%% Autor:    Fábio Leandro Rodrigues Cordeiro  (fabioleandro@pucminas.br)                            %% 
%% Co-autor: Prof. João Paulo Domingos Silva, Harison da Silva e Anderson Carvalho                   %%
%% Revisores normas NBR (Padrão PUC Minas): Helenice Rego Cunha e Prof. Theldo Cruz                  %%
%% Versão: 1.1     18 de dezembro 2015                     	                                     %%
%%%%%%%%%%%%%%%%%%%%%%%%%%%%%%%%%%%%%%%%%%%%%%%%%%%%%%%%%%%%%%%%%%%%%%%%%%%%%%%%%%%%%%%%%%%%%%%%%%%%%%%


\documentclass[a4paper,12pt,Times]{article}
\usepackage{abakos}  %pacote com padrão da Abakos baseado no padrão da PUC

%%%%%%%%%%%%%%%%%%%%%%%%%%%
%Capa da revista
%%%%%%%%%%%%%%%%%%%%%%%%%%

%\setcounter{page}{80} %iniciar contador de pagina de valor especificado
\newcommand{\monog}{Três Tipos Lógicos de Pesquisa}
\newcommand{\monogES}{Uma análise do discurso de Gilson Volpato}
\newcommand{\tipo}{Artigo }  % Especificar a seção tipo do trabalho: Artigo, Resumo, Tese, Dociê etc
\newcommand{\origem}{Brasil }
\newcommand{\editorial}{Belo Horizonte, p. 01-11, nov. 2015}  % p. xx-xx – páginas inicial-final do artigo
\newcommand{\lcc}{\scriptsize{Licença Creative Commons Attribution-NonCommercial-NoDerivs 3.0 Unported}}

%%%%%%%%%%%%%%%%%INFORMAÇÕES SOBRE AUTOR PRINCIPAL %%%%%%%%%%%%%%%%%%%%%%%%%%%%%%%
\newcommand{\AutorA}{Bernardo Ladeira Kartabil}
\newcommand{\funcaoA}{}
\newcommand{\emailA}{bernardo.kartabil@sga.pucminas.br}
\newcommand{\cursA}{Aluno do Programa de Graduação em Ciência da Computação}
% 
% Definir macros para o nome da Instituição, da Faculdade, etc.
\newcommand{\univ}{Pontifícia Universidade Católica de Minas Gerais}

\newcommand{\keyword}[1]{\textsf{#1}}

\begin{document}
% %%%%%%%%%%%%%%%%%%%%%%%%%%%%%%%%%%
% %% Pagina de titulo
% %%%%%%%%%%%%%%%%%%%%%%%%%%%%%%%%%%

\begin{center}
\includegraphics[scale=0.2]{figuras/brasao.jpg} \\
PONTIFÍCIA UNIVERSIDADE CATÓLICA DE MINAS GERAIS \\
Instituto de Ciências Exatas e de Informática

% \vspace{1.0cm}

\end{center}

 \vspace{0cm} {
 \singlespacing \Large{\monog \symbolfootnote[1]{Artigo apresentado ao Instituto de Ciências Exatas e Informática da Pontifícia Universidade Católica de Minas Gerais como pré-requisito para obtenção do título de Bacharel em Ciência da Computação.} \\ }
  \normalsize{\monogES}
 }

\vspace{1.0cm}

\begin{flushright}
\singlespacing 
\normalsize{\AutorA \footnote{\funcaoA \cursA, \origem -- \emailA . }} \\

%\normalsize{\AutorC \footnote{\funcaoC \cursC, \origem -- \emailC . }} \\
%\normalsize{\AutorD \footnote{\funcaD \cursD, \origem -- \emailD . }} \\
%deixar com o valor `0` e usar o '*' no inicio da frase
% \symbolfootnote[0]{Artigo recebido em 10 de julho de 1983 e aprovado em 29 de maio 2012}
\end{flushright}
\thispagestyle{empty}

\vspace{1.0cm}

\begin{abstract}
\noindent

O texto discute três tipos lógicos de pesquisa: descrição, associação e associação com interferência. Na descrição, uma variável é analisada isoladamente. Na associação, variáveis são relacionadas sem interferência direta. Já na associação com interferência, busca-se entender o mecanismo que une as variáveis.
\\\textbf{\keyword{Palavras-chave: }} Pesquisa. Associação. Interferência. Variáveis
\end{abstract}

%%%%%%%%%%%%%%%%%%%%%%%%%%%%%%%%%%%%%%%%%%%%%%%%%%%%%%%%%
 \newpage    %%%% CASO QUEIRA QUE O RESUMO FIQUE EM UMA PAGINA E O ABSTRACT EM OUTRA
\selectlanguage{english}
\begin{abstract}
\noindent
The text discusses three logical types of research: description, association and association with interference. In the description, a variable is analyzed in isolation. In association, variables are related without direct interference. In the association with interference, we seek to understand the mechanism that unites the variables.
\\\textbf{\keyword{Keywords: }} research. Association. Interference. Variables
\end{abstract}

\selectlanguage{brazilian}
 \onehalfspace  % espaçamento 1.5 entre linhas
 \setlength{\parindent}{1.25cm}

%%%%%%%%%%%%%%%%%%%%%%%%%%%%%%%%%%%%%%%%%%%%%%%%%
%% INICIO DO TEXTO
%%%%%%%%%%%%%%%%%%%%%%%%%%%%%%%%%%%%%%%%%%%%%%%%%

 %%%%%%%%%%%%%%%%%%%%%%%%%%%%%%%%%%%%%%%%%%%%%%%%%%%%%%%%%%%%%%%%%%%%%%%%%%%%%%%%%%%%%%%%%%%%%%%%%%%%%%%
%%%%%%%%%%%%%% Template de Artigo Adaptado para Trabalho de Diplomação do ICEI %%%%%%%%%%%%%%%%%%%%%%%%
%% codificação UTF-8 - Abntex - Latex -  							     %%
%% Autor:    Fábio Leandro Rodrigues Cordeiro  (fabioleandro@pucminas.br)                            %% 
%% Co-autores: Prof. João Paulo Domingos Silva, Harison da Silva e Anderson Carvalho		     %%
%% Revisores normas NBR (Padrão PUC Minas): Helenice Rego Cunha e Prof. Theldo Cruz                  %%
%% Versão: 1.1     18 de dezembro 2015                                                               %%
%%%%%%%%%%%%%%%%%%%%%%%%%%%%%%%%%%%%%%%%%%%%%%%%%%%%%%%%%%%%%%%%%%%%%%%%%%%%%%%%%%%%%%%%%%%%%%%%%%%%%%%
\section{\esp O que é uma hipótese e qual a sua importância no processo de pesquisa?
}

	Em primeiro lugar, precisamos entender quando se inicia uma pesquisa científica. O processo de pesquisa começa quando o ser humano se faz uma pergunta acerca
dos fenômenos que o rodeia e a sua resposta, quando não testada, é chamada de hipótese.Gilson Volpato exemplifica a temática com uma situação: "Quantas espécies
de aves ocorrem aqui?".Assim, o autor cria uma hipótese de que houve 20 espécies na região.Porém, após realizar uma análise de campo, constatou que na realidade 
houve apenas 8 espécies de fato na região, derrubando a hipótese antes criada. Seguindo a linha de raciocínio, o cientista afirma que o processo da pesquisa foi
condicionado pela pergunta, ou seja, a hipótese é um apenas elemento de "capricho acadêmico" e descartável nesse caso.Contudo, o internauta exemplifica uma situação
a qual a hipótese não é desnecessária: "Porque este carro não pega?". Primeiramente, a hipótese de que falta combustível no tanque é criada, mas logo é refutada através
de uma análise do composto que confirma ser gasolina. Assim, o cientista não tem a resposta, logo fica evidente a necessidade de elaborar uma segunda hipótese: "A bomba
de combustível está quebrada.".Ademais, é aconselhável a criação de uma terceira hipótese, porém relacionada à bomba de combustível, ou seja, é criado um aninhamento
de hipóteses que conduzem a sua pesquisa, tornando esse elemento essencial nesse processo.Além disso, Gilson Volpato cita o  "princípio da parcimônia", isto é, resolva
a questão mais simples para depois prosseguir para a mais complexa.

\section{\esp Tipos Lógicos de Pesquisa}

\subsection{\esp Lógica descritiva}

Portanto, qual é a finalidade da pesquisa sem hipótese ou aquela chamada de descritiva? Ela é feita com intuito de descrever situações ou ocorrências, ou seja, 
segundo Gilson Volpato,criam retrato das situações problema. Não adianta, segundo o autor, elaborar hióteses para ver o formato de uma célula, pois basta abrir o microscópio e ver de fato.Além disso, o internauta afirma que a pesquisa não se torna irrelevante por não ter hipótese, isso porque se os cientistas seguiram um conjunto de regras determinadas pela comunidade científica para sua elaboração, toda pesquisa tem o mesmo grau de qualidade.


\subsection{\esp Lógicas que relacionam uma ou mais variáveis}

Seguindo adiante na linha de raciocínio,Gilson Volpato explica outro tipo de pesquisa: "As pesquisas com hipótese.". Segundo o professor, as pesquisas com hipótese são aquelas que tratam a relação entre duas ou mais variáveis. Assim, Gilson divide essas relações em dois tipos: \textbf{"Associação e Interferência"}, caracterizando mais dois tipos lógicos de pesquisa.A pesquisa por associação ocorre quando uma ou mais variáveis se relacionam de forma que nenhuma se sobreponha à outra.Já a pesquisa com hipótese do tipo "interferência" é o contrário, ou seja, uma variável se sobrepõe à outra.O cientista cita um gráfico de população para exemplificar a 
pesquisa do primeiro tipo, já que as variáveis "número de habitantes" e "tempo" estão associadas sem sobreposição.Com relação à pesquisa do tipo "interferência",Gilson
afirma que existe uma relação de causa e efeito entre as variáveis analisadas.

\section{\esp CONCLUSÃO}

Então, é possível, com base no discurso de Gilson Volpato, afirmar que existem três tipos lógicos de pesquisa: A "descritiva", a "associativa" e a "interventiva
ou de interferência".A primeira não precisa obrigatoriamente de uma hipótese,pois serve para descrever as situações problema.A partir do segundo tipo, torna-se necessário
a elaboração de hipóteses e a análise da relação entre suas variáveis. Além disso, é importante colocar em pauta que esse conteúdo elaborado pelo cientista é atemporal,
isso porque pode ser aplicado para auxiliar na produção qualquer tipo de pesquisa científica.


   



\


%%%%%%%%%%%%%%%%%%%%%%%%%%%%%%%%%%%
%% FIM DO TEXTO
%%%%%%%%%%%%%%%%%%%%%%%%%%%%%%%%%%%

% \selectlanguage{brazil}
%%%%%%%%%%%%%%%%%%%%%%%%%%%%%%%%%%%
%% Inicio bibliografia
%%%%%%%%%%%%%%%%%%%%%%%%%%%%%%%%%%%

 \newpage
\singlespace{
\renewcommand\refname{REFERÊNCIAS}
\bibliographystyle{abntex2-alf}
\bibliography{bibliografia}

}

\end{document}


